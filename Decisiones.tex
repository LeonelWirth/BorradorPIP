    {\LARGE Decisiones}

    \begin{itemize}
        \item Evaluacion de implementar encoders en cuadratura: 
        
        Beneficios: implementar encoders de cuadratura nos brinda una resolucion en la medicion de velocidad.

        Requisitos para implementarlo: se debe conseguir encoders en cuadratura, tras una breve investigacion , se nota que solo se consiguen encoders de este tipo con funcionamiento coaxial al eje de rotacion. 
        Esto ultimo mencionado trae multiples desafios.
        El primero es la ubicacion de los encoders en el robot, una posibilidad analizada es acoplarlo con el eje paralelo y un arreglo de poleas y correas.
        El segundo es que para poder procesar 4 señales en cuadratura debemos cambiar el hardware principal por otro que tenga 4 modulos de cuadratura.

        Dicho esto, combinando las problematicas mecanicas asociadas en alineacion y complegidad estructural para implementar estos cambios, en complemento con que se debe hacer una completa reestructuracion de hardware, incluyendo el PCB y software; se suma a que el precio estimado de implementacion ronda los 25 a 30 mil pesos al dia 12/12/2021.
        Por esto se decide continuar con el encoder incremental que se posee actualmente.
        Para no quedarnos sin mejoras se decide a diseñar e implementar un disco de encoder con una mayor cantidad de ranuras para asi aumentar la resolucion de medicion.

        Falta agregar las tolerancias mecanicas de construccion.

        \item Evaluacion de reemplazo de bujes:
        
        Al momento de recibir el vehiculo, se nota que existe un amplio periodo de rotacion del eje del motor previo a que se mueva la rueda, partiendo de un estado estacionario a un estado en movimiento.

        Al desarmarlo, se identifica que el diametro interno de los bujes y el diametro externo del eje difieren en el orden del milimetro, esto sumado a que el tornilo pasante que sujeta tanto el eje con los bujes como el eje con la rueda, difieren de igual manera entre el diametro del agujero del eje y el diametro externo al tornillo.

        Al sumar el aporte de los 3 efectos en serie, se tiene un amplio "juego" en la rotacion de la rueda.

        Para intentar minimizarlo, se evaluacomprar ejes y tornillos pasantes que solucionen el problema encontrado. Luego de multiples busquedas y debido a que las medidas buscadas no son tamaños estandar, no se consigue un repuesto que se pueda comprar y resuelva directamente el problema encontrado.

        Posteriormente se evalua la fabricacion de los bujes necesarios, con un coste de 2000 pesos en materiales al mes de noviembre de 2021 y acceso a maquinaria pesada como torno y fresa, se decide implementar el cambio.

        El arreglo consistiria en fabricar nuevamente tanto bujes como ejes del vehiculo y realizar fijaciones con chavetas debido a las medidas no estandar que se poseen en las perforaciones del eje del motor.

        \item Metodo de identificacion del sistema:

        Inicialmente se procede a obtener el modelo matematico guiados por el estado del arte y papers de referencia.
        Inicialmente se opta por identificar el sistema, enviando una señal cuadrada de ciclo pseudoaleatorio con la herramienta MATLAB y devolverle al programa con el microcontrolador las mediciones de velocidad coorespondientes. Luego de esto, con este set de datos se busca estimar los paramtros del sistema (ya que conocemos como es la forma, variables y cantidad de polos y ceros) con herramientas propias de matlab.

        Se busca posteriormente pasar el sistema en variables de estado y controlarlo de esta manera. debido a que sabiamos el orden de algunos de los parametros del sistema por que los podiamos medir, observamos que la convergencia del algoritmo de matlab, nos daba un set de parametros valido, pero varios ordenes de magnitud mayores o menores a los valores que esperabamos. Por esto ultimo, y por el gran esfuerzo mecanico que le generaba la señal cuadrada pseudoaleatoria al sistema se opto por realizar los ensayos de los parametros de manera manual sin ayuda de matlab.

    \end{itemize}

