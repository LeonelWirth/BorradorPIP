\begin{titlepage}
    {\LARGE Objetivos}

  
    El objetivo de este trabajo es el diseño, desarrollo e implementación del sistema de control de movimiento para un robot 4WD (Four-Wheel Drive) capaz de recibir valores de referencia de velocidad deseados (o setpoints) para cada rueda por un protocolo de comunicación y establecerlos.
      
    
    Objetivos especificos:
    \begin{itemize}
        \item Caracterizar y modelar los cuatro sistemas (uno por cada rueda), compuesto por: motor, engranaje, rueda y encoder óptico.
        \item Diseñar y sintonizar un lazo de control de velocidad para cada rueda del robot de manera independiente.
        \item Utilizar un sistema operativo de tiempo real (RTOS) para la implementación de los cuatro lazos de control en un microcontrolador de la familia STM32. 
        \item Establecer la estructura de los paquetes de datos y el protocolo de comunicación entre un dispositivo externo (maestro), que enviará los setpoints, y el sistema a diseñar (esclavo) que recibirá las órdenes. La comunicación será a través de una interfaz serie UART.
        \item Realizar el control de versiones, y escribir la documentación, del código fuente del software desarrollado durante el proyecto.
        \item Diseñar e implementar una placa (PCB) que integre todo el hardware utilizado para el funcionamiento del sistema de control.
        \item Integrar todos los componentes anteriores sobre la plataforma móvil. El sistema resultante compuesto por: baterías, motores, encoders, sistema embebido e interfaz de comunicación es el sistema de control resultante. Realizar pruebas funcionales de todo el sistema.
        \item Verificar, mediante pruebas con diferentes cargas, la respuesta del control de lazo cerrado de cada rueda.
        \item Estimar el consumo energético del conjunto, evaluar e implementar las estrategias que correspondan para el uso de baterías de gel sobre el robot.
        \item Realizar pruebas en campo del sistema integrado con trayectorias predefinidas (líneas rectas y giros).

      \end{itemize}
\end{titlepage}
