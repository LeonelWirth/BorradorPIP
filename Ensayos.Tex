    {\LARGE Ensayos}

    \begin{itemize}
        \item Medicion de Ra:
        Ra [Ohm] es la resistencia ohmica del estator del motor. Para medirla se utiliza un multimetro [Uni-t UT890D+] y se mide directamente en bornes del motor multiples veces y luego se realiza un promedio de las mediciones.

        \item Ensayo para obtener La:
        La [Hy] representa inductancia del devanado estatorico del motor. Para la medicion de este parametro, primero se bloquea el rotor del motor para que no se genere una contra FEM, luego se coloca una resistencia conocida en serie con el motor y final mente se aplica un escalon de tension en bornes del motor. Se conecta un osciloscopio en bornes de la resistencia y se captura el transitorio del escalon, al conocer la resistencia y tener una imagen transitoria de la señal, se puede calcular la constante de tiempo del sistema, y con esta la inductancia La.

        \item Ensayo para obtener Kf:
        
        \item Ensayo para obtener Kt:
        
        \item Ensayo para obtener tao M:
        Tao M representa la constante de tiempo mecanica del sistema.
        
        \item Ensayo para obtener Tf:
        Tf representa el torque de friccion del sistema.
        
        \item Ensayo para obtener Iarr:
        Iarr representa la corriente de arranque del motor.
        
        \item Ensayo para obtener B:
        
        \item Esayo para obtener J: 
        J representa el momento de inercia.
        
        
    \end{itemize}

