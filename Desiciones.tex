\begin{titlepage}
    {\LARGE Desiciones}

    
    \begin{description}
        \item[Evaluacion de implementar encoders en cuadratura:] 
        Beneficios: implementar encoders de cuadratura nos brinda una resolucion en la medicion de velocidad.

        Requisitos para implementarlo: se debe conseguir encoders en cuadratura, tras una breve investigacion , se nota que solo se consiguen encoders de este tipo con funcionamiento coaxial al eje de rotacion. 
        Esto ultimo mencionado trae multiples desafios.
        El primero es la ubicacion de los encoders en el robot, una posibilidad analizada es acoplarlo con el eje paralelo y un arreglo de poleas y correas.
        El segundo es que para poder procesar 4 señales en cuadratura debemos cambiar el hardware principal por otro que tenga 4 modulos de cuadratura.

        Dicho esto, combinando las problematicas mecanicas asociadas en alineacion y complegidad estructural para implementar estos cambios, en complemento con que se debe hacer una completa reestructuracion de hardware, incluyendo el PCB y software; se suma a que el precio estimado de implementacion ronda los 25 a 30 mil pesos al dia 12/12/2021.
        Por esto se decide continuar con el encoder incremental que se posee actualmente.
        Para no quedarnos sin mejoras se decide a diseñar e implementar un disco de encoder con una mayor cantidad de ranuras para asi aumentar la resolucion de medicion.

        Falta agregar las tolerancias mecanicas de construccion.
        \item[] 
        \item[] 
    \end{description}
\end{titlepage}
